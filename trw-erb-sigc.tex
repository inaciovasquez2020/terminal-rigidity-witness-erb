\documentclass[11pt,a4paper]{article}

\usepackage[utf8]{inputenc}
\usepackage[T1]{fontenc}
\usepackage[margin=1in]{geometry}
\usepackage{amsmath,amssymb,amsthm,mathtools}
\usepackage{hyperref}
\usepackage{enumitem}
\usepackage{microtype}
\usepackage{booktabs}

\title{Terminal Rigidity Witnesses, Zero Channels, and the SIGC Functor}
\author{Inacio F. Vasquez\\Independent Researcher}
\date{\today}

\newtheorem{theorem}{Theorem}
\newtheorem{lemma}{Lemma}
\newtheorem{definition}{Definition}
\newtheorem{corollary}{Corollary}
\newtheorem{axiom}{Axiom}

\renewcommand{\Cap}{\mathrm{Cap}}
\newcommand{\TC}{\mathrm{TC}}
\newcommand{\ED}{\mathrm{ED}}
\newcommand{\SIGC}{\mathbf{SIGC}}
\newcommand{\PhysSys}{\mathbf{PhysSys}}

\begin{document}
\maketitle

\begin{abstract}
We formalize the Einstein--Rosen bridge (ERB) as a \emph{Terminal Rigidity Witness} within the Unified Rigidity Framework (URF).  
We define a minimal category of physical systems, construct the Terminal Witness Functor into the SIGC preorder, derive the zero-capacity property of ERB from topological censorship, and integrate black hole information scenarios as explicit capacity-regime distinctions.  
This provides a structural, non-speculative account of geometric objects with maximal global structure but zero operational information capacity.
\end{abstract}

\section{Motivation and Background}

Modern discussions of wormholes, entanglement, and black hole reconstruction frequently blur the distinction between \emph{geometric structure} and \emph{operational information}.  
The Unified Rigidity Framework separates these notions by imposing explicit capacity accounting: only information transmitted through admissible physical interfaces contributes to operational content.

The Einstein--Rosen bridge is the canonical example of a system with rich global geometry but no transmissible information.  
We show that ERB is the universal zero object in the SIGC preorder and serves as a terminal physical witness for rigidity.

\section{The Category of Physical Systems}

\begin{definition}[Physical System]
A physical system is a triple
\[
S := (\mathcal{I}, \mathcal{L}, E)
\]
where:
\begin{itemize}
\item $\mathcal{I}$ is a finite set of admissible interfaces,
\item $\mathcal{L} = (Y_{1:T})$ is the observable interaction log,
\item $E = \sum_{t=1}^T \Delta Q_t$ is total dissipated energy.
\end{itemize}
\end{definition}

Each system induces a channel
\[
\mathcal{C}_S : X \to Y_{1:T}.
\]

\begin{definition}[Morphisms]
A morphism $f:S\to S'$ is a triple $(f_I,f_L,f_E)$ satisfying
\[
\mathcal{C}_{S'} = f_L \circ \mathcal{C}_S \circ f_I,
\qquad E' \ge E.
\]
\end{definition}

This defines a category $\PhysSys$.

\section{SIGC and Channel Equivalence}

Let $\SIGC$ be the preorder of channel equivalence classes modulo admissible simulation:
\[
\mathcal{C}_1 \sim \mathcal{C}_2 \iff 
\mathcal{C}_1 = g \circ \mathcal{C}_2 \circ f
\]
for admissible $f,g$.

\begin{definition}[Capacity Preorder]
\[
\mathcal{C}_1 \preceq \mathcal{C}_2 \iff 
\Cap(\mathcal{C}_1) \le \Cap(\mathcal{C}_2).
\]
\end{definition}

\section{Terminal Witness Functor}

\begin{definition}[Terminal Witness Functor]
\[
\mathcal{T} : \PhysSys \to \SIGC,
\qquad
\mathcal{T}(S) := [\mathcal{C}_S]_{\sim}.
\]
\end{definition}

\begin{theorem}[Well-Definedness]
If $S \cong S'$ in $\PhysSys$, then $\mathcal{T}(S)=\mathcal{T}(S')$.
\end{theorem}

\section{ERB as Zero Object}

\begin{definition}[TRW--ERB]
Let $(M,g)$ be the maximal analytic Schwarzschild extension and
\[
\mathcal{C}_{\mathrm{ERB}} : \Sigma_- \to \Sigma_+
\]
the induced causal channel. Define
\[
\mathbf{0}_{\SIGC} := [\mathcal{C}_{\mathrm{ERB}}].
\]
\end{definition}

\begin{lemma}[Topological Censorship Channel Nullity]
Assuming ANEC and global hyperbolicity,
\[
\forall X,\quad I(X;Y)=0.
\]
Hence
\[
\Cap(\mathbf{0}_{\SIGC}) = 0.
\]
\end{lemma}

\begin{proof}
Topological censorship implies no causal curve connects $\Sigma_-$ to $\Sigma_+$.  
Therefore outputs are independent of inputs and mutual information vanishes.
\end{proof}

\begin{corollary}
For any channel $\mathcal{C}$,
\[
\Cap(\mathcal{C})=0 \iff \mathcal{C}\sim \mathbf{0}_{\SIGC}.
\]
\end{corollary}

\section{Black Hole Information Paradox}

We distinguish two regimes.

\subsection{Refinement-Only Regime}
Observer restricted to physical interfaces:
\[
I_{\mathrm{int}} = 0,
\qquad
\Cap_{\mathrm{ext}} \le \SIGC(E).
\]

\subsection{Global-Invariant Regime}
Observer claims access to global purification map $\mathcal{G}$:
\[
I_{\mathrm{int}} > 0
\]
requires
\[
\SIGC_{\mathcal{G}} \ge H(\text{interior}).
\]

\begin{axiom}[BHIP Admissibility]
Global reconstruction is admissible iff its SIGC cost is explicitly accounted.
\end{axiom}

\section{Preorder as Channel Simulation}

\begin{theorem}
\[
\mathcal{C}_1 \preceq \mathcal{C}_2
\iff
\exists f,g:\ \mathcal{C}_1 = g\circ\mathcal{C}_2\circ f.
\]
\end{theorem}

\begin{corollary}
\[
\mathcal{T}(S_1) \preceq \mathcal{T}(S_2)
\iff
S_1 \text{ reduces to } S_2 \text{ by admissible refinement}.
\]
\end{corollary}

\section{Certificate Schema (JSON)}

Every admissibility certificate must include:
\begin{verbatim}
{
  "baseline_zero_object": "TRW-ERB",
  "baseline_checks": {
    "cap_zero": true,
    "tc_zero": true,
    "ed_zero": true
  },
  "zero_equivalence": {
    "definition": "Cap(C)=0 iff C ~ zeroSIGC",
    "verified": true
  }
}
\end{verbatim}

\section{Conclusion}

ERB is the universal terminal witness of operational rigidity:  
a system with maximal geometric structure and zero information capacity.  
All admissible physical systems are functorially mapped into the SIGC preorder,
with ERB as the unique minimal element.
\begin{thebibliography}{99}

\bibitem{EinsteinRosen1935}
A. Einstein and N. Rosen,
\emph{The Particle Problem in the General Theory of Relativity},
Physical Review \textbf{48}, 73--77 (1935).

\bibitem{Kruskal1960}
M. D. Kruskal,
\emph{Maximal Extension of Schwarzschild Metric},
Physical Review \textbf{119}, 1743--1745 (1960).

\bibitem{FriedmanSchleichWitt1993}
J. L. Friedman, K. Schleich, and D. M. Witt,
\emph{Topological Censorship},
Physical Review Letters \textbf{71}, 1486--1489 (1993).

\bibitem{SorkinWoolgar1996}
R. D. Sorkin and E. Woolgar,
\emph{A Causal Order for Spacetimes with Cauchy Horizons},
Classical and Quantum Gravity \textbf{13}, 1971--1993 (1996).

\bibitem{MorrisThorne1988}
M. Morris and K. Thorne,
\emph{Wormholes in Spacetime and Their Use for Interstellar Travel},
American Journal of Physics \textbf{56}, 395--412 (1988).

\bibitem{Visser1996}
M. Visser,
\emph{Lorentzian Wormholes: From Einstein to Hawking},
Springer (1996).

\bibitem{Hawking1976}
S. Hawking,
\emph{Breakdown of Predictability in Gravitational Collapse},
Physical Review D \textbf{14}, 2460--2473 (1976).

\bibitem{AMPS2013}
A. Almheiri, D. Marolf, J. Polchinski, and J. Sully,
\emph{Black Holes: Complementarity or Firewalls?},
Journal of High Energy Physics \textbf{2013}, 62.

\bibitem{MaldacenaSusskind2013}
J. Maldacena and L. Susskind,
\emph{Cool Horizons for Entangled Black Holes},
Fortschritte der Physik \textbf{61}, 781--811 (2013).

\bibitem{GaoJafferisWall2017}
J. Gao, D. Jafferis, and A. Wall,
\emph{Traversable Wormholes via Quantum Effects},
Physical Review Letters \textbf{118}, 211601 (2017).

\bibitem{Shannon1948}
C. Shannon,
\emph{A Mathematical Theory of Communication},
Bell System Technical Journal \textbf{27}, 379--423 (1948).

\bibitem{Landauer1961}
R. Landauer,
\emph{Irreversibility and Heat Generation in the Computing Process},
IBM Journal of Research and Development \textbf{5}, 183--191 (1961).

\bibitem{Libkin2004}
L. Libkin,
\emph{Elements of Finite Model Theory},
Springer (2004).

\bibitem{Grohe2017}
M. Grohe,
\emph{Descriptive Complexity, Canonisation, and Definable Graph Structure Theory},
Cambridge University Press (2017).

\bibitem{DawarGrohe2009}
A. Dawar and M. Grohe,
\emph{The Descriptive Complexity of Counting},
Theoretical Computer Science \textbf{410}, 199--217 (2009).

\bibitem{MacLane1971}
S. Mac Lane,
\emph{Categories for the Working Mathematician},
Springer (1971).

\end{thebibliography}

\end{document}

